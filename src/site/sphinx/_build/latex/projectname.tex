% Generated by Sphinx.
\def\sphinxdocclass{report}
\documentclass[letterpaper,10pt,english]{sphinxmanual}
\usepackage[utf8]{inputenc}
\DeclareUnicodeCharacter{00A0}{\nobreakspace}
\usepackage{cmap}
\usepackage[T1]{fontenc}
\usepackage{babel}
\usepackage{times}
\usepackage[Bjarne]{fncychap}
\usepackage{longtable}
\usepackage{sphinx}
\usepackage{multirow}


\title{${project.name}}
\date{February 17, 2015}
\release{${project.version}}
\author{}
\newcommand{\sphinxlogo}{}
\renewcommand{\releasename}{Release}
\makeindex

\makeatletter
\def\PYG@reset{\let\PYG@it=\relax \let\PYG@bf=\relax%
    \let\PYG@ul=\relax \let\PYG@tc=\relax%
    \let\PYG@bc=\relax \let\PYG@ff=\relax}
\def\PYG@tok#1{\csname PYG@tok@#1\endcsname}
\def\PYG@toks#1+{\ifx\relax#1\empty\else%
    \PYG@tok{#1}\expandafter\PYG@toks\fi}
\def\PYG@do#1{\PYG@bc{\PYG@tc{\PYG@ul{%
    \PYG@it{\PYG@bf{\PYG@ff{#1}}}}}}}
\def\PYG#1#2{\PYG@reset\PYG@toks#1+\relax+\PYG@do{#2}}

\expandafter\def\csname PYG@tok@gu\endcsname{\def\PYG@tc##1{\textcolor[rgb]{0.67,0.67,0.67}{##1}}}
\expandafter\def\csname PYG@tok@gt\endcsname{\def\PYG@tc##1{\textcolor[rgb]{0.67,0.00,0.00}{##1}}}
\expandafter\def\csname PYG@tok@gs\endcsname{\let\PYG@bf=\textbf}
\expandafter\def\csname PYG@tok@gr\endcsname{\def\PYG@tc##1{\textcolor[rgb]{0.67,0.00,0.00}{##1}}}
\expandafter\def\csname PYG@tok@cm\endcsname{\let\PYG@it=\textit\def\PYG@tc##1{\textcolor[rgb]{0.60,0.60,0.53}{##1}}}
\expandafter\def\csname PYG@tok@vg\endcsname{\def\PYG@tc##1{\textcolor[rgb]{0.00,0.50,0.50}{##1}}}
\expandafter\def\csname PYG@tok@m\endcsname{\def\PYG@tc##1{\textcolor[rgb]{0.00,0.60,0.60}{##1}}}
\expandafter\def\csname PYG@tok@mh\endcsname{\def\PYG@tc##1{\textcolor[rgb]{0.00,0.60,0.60}{##1}}}
\expandafter\def\csname PYG@tok@go\endcsname{\def\PYG@tc##1{\textcolor[rgb]{0.53,0.53,0.53}{##1}}}
\expandafter\def\csname PYG@tok@ge\endcsname{\let\PYG@it=\textit}
\expandafter\def\csname PYG@tok@gd\endcsname{\def\PYG@tc##1{\textcolor[rgb]{0.00,0.00,0.00}{##1}}\def\PYG@bc##1{\setlength{\fboxsep}{0pt}\colorbox[rgb]{1.00,0.87,0.87}{\strut ##1}}}
\expandafter\def\csname PYG@tok@il\endcsname{\def\PYG@tc##1{\textcolor[rgb]{0.00,0.60,0.60}{##1}}}
\expandafter\def\csname PYG@tok@cs\endcsname{\let\PYG@bf=\textbf\let\PYG@it=\textit\def\PYG@tc##1{\textcolor[rgb]{0.60,0.60,0.60}{##1}}}
\expandafter\def\csname PYG@tok@cp\endcsname{\let\PYG@bf=\textbf\def\PYG@tc##1{\textcolor[rgb]{0.60,0.60,0.60}{##1}}}
\expandafter\def\csname PYG@tok@gi\endcsname{\def\PYG@tc##1{\textcolor[rgb]{0.00,0.00,0.00}{##1}}\def\PYG@bc##1{\setlength{\fboxsep}{0pt}\colorbox[rgb]{0.87,1.00,0.87}{\strut ##1}}}
\expandafter\def\csname PYG@tok@gh\endcsname{\def\PYG@tc##1{\textcolor[rgb]{0.60,0.60,0.60}{##1}}}
\expandafter\def\csname PYG@tok@ni\endcsname{\def\PYG@tc##1{\textcolor[rgb]{0.50,0.00,0.50}{##1}}}
\expandafter\def\csname PYG@tok@nn\endcsname{\def\PYG@tc##1{\textcolor[rgb]{0.33,0.33,0.33}{##1}}}
\expandafter\def\csname PYG@tok@no\endcsname{\def\PYG@tc##1{\textcolor[rgb]{0.00,0.50,0.50}{##1}}}
\expandafter\def\csname PYG@tok@na\endcsname{\def\PYG@tc##1{\textcolor[rgb]{0.00,0.50,0.50}{##1}}}
\expandafter\def\csname PYG@tok@nb\endcsname{\def\PYG@tc##1{\textcolor[rgb]{0.60,0.60,0.60}{##1}}}
\expandafter\def\csname PYG@tok@nc\endcsname{\let\PYG@bf=\textbf\def\PYG@tc##1{\textcolor[rgb]{0.27,0.33,0.53}{##1}}}
\expandafter\def\csname PYG@tok@ne\endcsname{\let\PYG@bf=\textbf\def\PYG@tc##1{\textcolor[rgb]{0.60,0.00,0.00}{##1}}}
\expandafter\def\csname PYG@tok@nf\endcsname{\let\PYG@bf=\textbf\def\PYG@tc##1{\textcolor[rgb]{0.60,0.00,0.00}{##1}}}
\expandafter\def\csname PYG@tok@si\endcsname{\def\PYG@tc##1{\textcolor[rgb]{0.73,0.53,0.27}{##1}}}
\expandafter\def\csname PYG@tok@s2\endcsname{\def\PYG@tc##1{\textcolor[rgb]{0.73,0.53,0.27}{##1}}}
\expandafter\def\csname PYG@tok@vi\endcsname{\def\PYG@tc##1{\textcolor[rgb]{0.00,0.50,0.50}{##1}}}
\expandafter\def\csname PYG@tok@nt\endcsname{\def\PYG@tc##1{\textcolor[rgb]{0.00,0.00,0.50}{##1}}}
\expandafter\def\csname PYG@tok@nv\endcsname{\def\PYG@tc##1{\textcolor[rgb]{0.00,0.50,0.50}{##1}}}
\expandafter\def\csname PYG@tok@s1\endcsname{\def\PYG@tc##1{\textcolor[rgb]{0.73,0.53,0.27}{##1}}}
\expandafter\def\csname PYG@tok@vc\endcsname{\def\PYG@tc##1{\textcolor[rgb]{0.00,0.50,0.50}{##1}}}
\expandafter\def\csname PYG@tok@gp\endcsname{\def\PYG@tc##1{\textcolor[rgb]{0.33,0.33,0.33}{##1}}}
\expandafter\def\csname PYG@tok@sh\endcsname{\def\PYG@tc##1{\textcolor[rgb]{0.73,0.53,0.27}{##1}}}
\expandafter\def\csname PYG@tok@ow\endcsname{\let\PYG@bf=\textbf}
\expandafter\def\csname PYG@tok@sx\endcsname{\def\PYG@tc##1{\textcolor[rgb]{0.73,0.53,0.27}{##1}}}
\expandafter\def\csname PYG@tok@bp\endcsname{\def\PYG@tc##1{\textcolor[rgb]{0.60,0.60,0.60}{##1}}}
\expandafter\def\csname PYG@tok@c1\endcsname{\let\PYG@it=\textit\def\PYG@tc##1{\textcolor[rgb]{0.60,0.60,0.53}{##1}}}
\expandafter\def\csname PYG@tok@kc\endcsname{\let\PYG@bf=\textbf}
\expandafter\def\csname PYG@tok@c\endcsname{\let\PYG@it=\textit\def\PYG@tc##1{\textcolor[rgb]{0.60,0.60,0.53}{##1}}}
\expandafter\def\csname PYG@tok@mf\endcsname{\def\PYG@tc##1{\textcolor[rgb]{0.00,0.60,0.60}{##1}}}
\expandafter\def\csname PYG@tok@err\endcsname{\def\PYG@tc##1{\textcolor[rgb]{0.65,0.09,0.09}{##1}}\def\PYG@bc##1{\setlength{\fboxsep}{0pt}\colorbox[rgb]{0.89,0.82,0.82}{\strut ##1}}}
\expandafter\def\csname PYG@tok@mb\endcsname{\def\PYG@tc##1{\textcolor[rgb]{0.00,0.60,0.60}{##1}}}
\expandafter\def\csname PYG@tok@ss\endcsname{\def\PYG@tc##1{\textcolor[rgb]{0.73,0.53,0.27}{##1}}}
\expandafter\def\csname PYG@tok@sr\endcsname{\def\PYG@tc##1{\textcolor[rgb]{0.50,0.50,0.00}{##1}}}
\expandafter\def\csname PYG@tok@mo\endcsname{\def\PYG@tc##1{\textcolor[rgb]{0.00,0.60,0.60}{##1}}}
\expandafter\def\csname PYG@tok@kd\endcsname{\let\PYG@bf=\textbf}
\expandafter\def\csname PYG@tok@mi\endcsname{\def\PYG@tc##1{\textcolor[rgb]{0.00,0.60,0.60}{##1}}}
\expandafter\def\csname PYG@tok@kn\endcsname{\let\PYG@bf=\textbf}
\expandafter\def\csname PYG@tok@o\endcsname{\let\PYG@bf=\textbf}
\expandafter\def\csname PYG@tok@kr\endcsname{\let\PYG@bf=\textbf}
\expandafter\def\csname PYG@tok@s\endcsname{\def\PYG@tc##1{\textcolor[rgb]{0.73,0.53,0.27}{##1}}}
\expandafter\def\csname PYG@tok@kp\endcsname{\let\PYG@bf=\textbf}
\expandafter\def\csname PYG@tok@w\endcsname{\def\PYG@tc##1{\textcolor[rgb]{0.73,0.73,0.73}{##1}}}
\expandafter\def\csname PYG@tok@kt\endcsname{\let\PYG@bf=\textbf\def\PYG@tc##1{\textcolor[rgb]{0.27,0.33,0.53}{##1}}}
\expandafter\def\csname PYG@tok@sc\endcsname{\def\PYG@tc##1{\textcolor[rgb]{0.73,0.53,0.27}{##1}}}
\expandafter\def\csname PYG@tok@sb\endcsname{\def\PYG@tc##1{\textcolor[rgb]{0.73,0.53,0.27}{##1}}}
\expandafter\def\csname PYG@tok@k\endcsname{\let\PYG@bf=\textbf}
\expandafter\def\csname PYG@tok@se\endcsname{\def\PYG@tc##1{\textcolor[rgb]{0.73,0.53,0.27}{##1}}}
\expandafter\def\csname PYG@tok@sd\endcsname{\def\PYG@tc##1{\textcolor[rgb]{0.73,0.53,0.27}{##1}}}

\def\PYGZbs{\char`\\}
\def\PYGZus{\char`\_}
\def\PYGZob{\char`\{}
\def\PYGZcb{\char`\}}
\def\PYGZca{\char`\^}
\def\PYGZam{\char`\&}
\def\PYGZlt{\char`\<}
\def\PYGZgt{\char`\>}
\def\PYGZsh{\char`\#}
\def\PYGZpc{\char`\%}
\def\PYGZdl{\char`\$}
\def\PYGZhy{\char`\-}
\def\PYGZsq{\char`\'}
\def\PYGZdq{\char`\"}
\def\PYGZti{\char`\~}
% for compatibility with earlier versions
\def\PYGZat{@}
\def\PYGZlb{[}
\def\PYGZrb{]}
\makeatother

\renewcommand\PYGZsq{\textquotesingle}

\begin{document}

\maketitle
\tableofcontents
\phantomsection\label{index::doc}


The \emph{sphinx-maven} plugin is a \href{http://maven.apache.org/plugins/maven-site-plugin/}{Maven site plugin} that uses \href{http://sphinx.pocoo.org/}{Sphinx} to generate the main documentation.
Sphinx itself was origially created by the Python community for the new Python documentation. It uses a
plain text format called \href{http://docutils.sf.net/rst.html}{reStructured Text} which it then compiles into a variety of documentation formats
such as HTML, LaTeX (for PDF), epub. reStructured Text is similar to \href{http://daringfireball.net/projects/markdown/}{Markdown} but - at least via Sphinx -
has better support for multi-page documentation.

The \emph{sphinx-maven} plugin is BSD licensed just as Sphinx itself is.


\chapter{Creating the documentation}
\label{basic-usage:introduction}\label{basic-usage:creating-the-documentation}\label{basic-usage:markdown}\label{basic-usage::doc}\label{basic-usage:contents}
First, create a folder \code{src/site/sphinx}. This folder will contain the \href{http://docutils.sf.net/rst.html}{reStructured Text} source files plus
any additional things like themes and configuration. The name of the folder can be changed via options should
you want a different folder.

Next, add the documentation. The \href{http://sphinx.pocoo.org/tutorial.html}{Sphinx first steps tutorial} gives a good introduction into the required
tasks. Basically what you need is
\begin{itemize}
\item {} 
A configuration file called \href{http://sphinx.pocoo.org/config.html}{conf.py} that defines the theme and other options (such as which output formats etc.)

\item {} 
The documentation files in reStructured Text format.

\item {} 
Additional files such as static files (images etc.), usually in a \code{\_static} sub directory.

\item {} 
Optionally, a customized theme in a sub directory called \code{\_theme}

\end{itemize}

For good examples of documentation, see \href{http://sphinx.pocoo.org/examples.html}{Sphinx' examples page}.  The documentation for this plugin itself is
based on the documentation for \href{http://werkzeug.pocoo.org/docs/}{Werkzeug} (documentation source for it can be found on \href{https://github.com/mitsuhiko/werkzeug/tree/master/docs}{Werkzeug's github page})
and \href{http://docs.celeryproject.org/en/latest/index.html}{Celery} (documentation source can be found on \href{http://docs.celeryproject.org/en/latest/index.html}{Celery's github page}).


\chapter{Running as part of the \texttt{site} lifecycle}
\label{basic-usage:running-as-part-of-the-site-lifecycle}
Simply add the sphinx maven plugin to your \code{pom.xml}:

\begin{Verbatim}[commandchars=\\\{\}]
\PYGZlt{}reporting\PYGZgt{}
  \PYGZlt{}plugins\PYGZgt{}
    \PYGZlt{}plugin\PYGZgt{}
      \PYGZlt{}groupId\PYGZgt{}org.apache.maven.plugins\PYGZlt{}/groupId\PYGZgt{}
      \PYGZlt{}artifactId\PYGZgt{}maven\PYGZhy{}project\PYGZhy{}info\PYGZhy{}reports\PYGZhy{}plugin\PYGZlt{}/artifactId\PYGZgt{}
      \PYGZlt{}version\PYGZgt{}2.4\PYGZlt{}/version\PYGZgt{}
      \PYGZlt{}reportSets\PYGZgt{}
        \PYGZlt{}reportSet\PYGZgt{}
          \PYGZlt{}reports\PYGZgt{}\PYGZlt{}/reports\PYGZgt{}
        \PYGZlt{}/reportSet\PYGZgt{}
      \PYGZlt{}/reportSets\PYGZgt{}
    \PYGZlt{}/plugin\PYGZgt{}
    \PYGZlt{}plugin\PYGZgt{}
      \PYGZlt{}groupId\PYGZgt{}org.tomdz.maven\PYGZlt{}/groupId\PYGZgt{}
      \PYGZlt{}artifactId\PYGZgt{}sphinx\PYGZhy{}maven\PYGZhy{}plugin\PYGZlt{}/artifactId\PYGZgt{}
      \PYGZlt{}version\PYGZgt{}1.0.1\PYGZlt{}/version\PYGZgt{}
    \PYGZlt{}/plugin\PYGZgt{}
  \PYGZlt{}/plugins\PYGZgt{}
\PYGZlt{}/reporting\PYGZgt{}
\end{Verbatim}

It is important that you set the \code{reportSet} attribute of the \code{project-info-reports} plugin to an empty set of
\code{reports}. If not then the default \code{about} report will be generated which conflicts with the \code{sphinx-maven}
plugin, and in effect Sphinx will not be run.

\emph{Maven 3} changes how reporting plugins are specified. A \code{profile} can be used to define a \code{pom.xml} that can
be used with both Maven 2 and Maven 3:

\begin{Verbatim}[commandchars=\\\{\}]
\PYGZlt{}profiles\PYGZgt{}
  \PYGZlt{}profile\PYGZgt{}
    \PYGZlt{}id\PYGZgt{}maven\PYGZhy{}3\PYGZlt{}/id\PYGZgt{}
    \PYGZlt{}activation\PYGZgt{}
      \PYGZlt{}file\PYGZgt{}
         \PYGZlt{}!\PYGZhy{}\PYGZhy{}  This employs that the basedir expression is only recognized by Maven 3.x (see MNG\PYGZhy{}2363) \PYGZhy{}\PYGZhy{}\PYGZgt{}
        \PYGZlt{}exists\PYGZgt{}\PYGZdl{}\PYGZob{}basedir\PYGZcb{}\PYGZlt{}/exists\PYGZgt{}
      \PYGZlt{}/file\PYGZgt{}
    \PYGZlt{}/activation\PYGZgt{}
    \PYGZlt{}build\PYGZgt{}
      \PYGZlt{}plugins\PYGZgt{}
        \PYGZlt{}plugin\PYGZgt{}
          \PYGZlt{}groupId\PYGZgt{}org.apache.maven.plugins\PYGZlt{}/groupId\PYGZgt{}
          \PYGZlt{}artifactId\PYGZgt{}maven\PYGZhy{}site\PYGZhy{}plugin\PYGZlt{}/artifactId\PYGZgt{}
          \PYGZlt{}version\PYGZgt{}3.0\PYGZlt{}/version\PYGZgt{}
          \PYGZlt{}configuration\PYGZgt{}
            \PYGZlt{}reportPlugins\PYGZgt{}
              \PYGZlt{}plugin\PYGZgt{}
                \PYGZlt{}groupId\PYGZgt{}org.apache.maven.plugins\PYGZlt{}/groupId\PYGZgt{}
                \PYGZlt{}artifactId\PYGZgt{}maven\PYGZhy{}project\PYGZhy{}info\PYGZhy{}reports\PYGZhy{}plugin\PYGZlt{}/artifactId\PYGZgt{}
                \PYGZlt{}version\PYGZgt{}2.4\PYGZlt{}/version\PYGZgt{}
                \PYGZlt{}reportSets\PYGZgt{}
                  \PYGZlt{}reportSet\PYGZgt{}
                    \PYGZlt{}reports\PYGZgt{}\PYGZlt{}/reports\PYGZgt{}
                  \PYGZlt{}/reportSet\PYGZgt{}
                \PYGZlt{}/reportSets\PYGZgt{}
              \PYGZlt{}/plugin\PYGZgt{}
              \PYGZlt{}plugin\PYGZgt{}
                \PYGZlt{}groupId\PYGZgt{}org.tomdz.maven\PYGZlt{}/groupId\PYGZgt{}
                \PYGZlt{}artifactId\PYGZgt{}sphinx\PYGZhy{}maven\PYGZhy{}plugin\PYGZlt{}/artifactId\PYGZgt{}
                \PYGZlt{}version\PYGZgt{}1.0.1\PYGZlt{}/version\PYGZgt{}
              \PYGZlt{}/plugin\PYGZgt{}
            \PYGZlt{}/reportPlugins\PYGZgt{}
          \PYGZlt{}/configuration\PYGZgt{}
        \PYGZlt{}/plugin\PYGZgt{}
      \PYGZlt{}/plugins\PYGZgt{}
    \PYGZlt{}/build\PYGZgt{}
  \PYGZlt{}/profile\PYGZgt{}
\PYGZlt{}/profiles\PYGZgt{}
\end{Verbatim}

The profile will only be activated if Maven 3 is used to generate the site. For more details about Maven 3
and the site plugin see the \href{https://cwiki.apache.org/MAVEN/maven-3x-and-site-plugin.html}{Maven 3 site plugin wiki page} and this \href{http://whatiscomingtomyhead.wordpress.com/2011/06/05/maven-3-site-plugin-how-to/}{Maven 3 site plugin howto}.

Now all you need to do is to generate the documentation:

\begin{Verbatim}[commandchars=\\\{\}]
mvn site
\end{Verbatim}

This will generate the documentation in the \emph{target/site} folder.


\chapter{Running as part of the normal lifecycle}
\label{basic-usage:running-as-part-of-the-normal-lifecycle}
You can also bind the plugin to a normal lifecycle phase. This is for instance useful if you want to generate a
documentation artifact and deploy it somewhere.

The plugin configuration is pretty much the same, the only difference is that you need to add an \code{execution} section.
It might also be useful to change the \code{outputDirectory} to a different folder as the plugin by default puts the
generated documentation into the \code{target/site} folder.

A sample \code{pom.xml} plugin section could look like this:

\begin{Verbatim}[commandchars=\\\{\}]
\PYGZlt{}build\PYGZgt{}
  \PYGZlt{}plugins\PYGZgt{}
    ...
    \PYGZlt{}plugin\PYGZgt{}
      \PYGZlt{}groupId\PYGZgt{}org.tomdz.maven\PYGZlt{}/groupId\PYGZgt{}
      \PYGZlt{}artifactId\PYGZgt{}sphinx\PYGZhy{}maven\PYGZhy{}plugin\PYGZlt{}/artifactId\PYGZgt{}
      \PYGZlt{}version\PYGZgt{}1.0.1\PYGZlt{}/version\PYGZgt{}
      \PYGZlt{}configuration\PYGZgt{}
        \PYGZlt{}outputDirectory\PYGZgt{}\PYGZdl{}\PYGZob{}project.build.directory\PYGZcb{}/docs\PYGZlt{}/outputDirectory\PYGZgt{}
      \PYGZlt{}/configuration\PYGZgt{}
      \PYGZlt{}executions\PYGZgt{}
        \PYGZlt{}execution\PYGZgt{}
          \PYGZlt{}phase\PYGZgt{}package\PYGZlt{}/phase\PYGZgt{}
          \PYGZlt{}goals\PYGZgt{}
            \PYGZlt{}goal\PYGZgt{}generate\PYGZlt{}/goal\PYGZgt{}
          \PYGZlt{}/goals\PYGZgt{}
        \PYGZlt{}/execution\PYGZgt{}
      \PYGZlt{}/executions\PYGZgt{}
    \PYGZlt{}/plugin\PYGZgt{}
    ...
  \PYGZlt{}/plugins\PYGZgt{}
\PYGZlt{}/build\PYGZgt{}
\end{Verbatim}


\chapter{Configuration}
\label{configuration:rst2pdf-manual}\label{configuration:configuration}\label{configuration::doc}
The \code{sphinx-maven} plugin has these configuration options:

\begin{tabulary}{\linewidth}{|L|L|L|}
\hline
\textsf{\relax 
Parameter
} & \textsf{\relax 
Description
} & \textsf{\relax 
Default value
}\\
\hline
\code{sourceDirectory}
 & 
The directory containing the documentation source.
 & 
\code{\$\{basedir\}/src/site/sphinx}
\\

\code{outputDirectory}
 & 
The directory where the generated output will be placed.
 & 
\code{\$\{project.reporting.outputDirectory\}}
\\

\code{fork}
 & 
Whether to run Sphinx in a forked JVM instance.
 & 
\code{false}
\\

\code{jvm}
 & 
The JVM binary to use. If not set, then will use the one used to run the plugin.
 & \\

\code{argLine}
 & 
Additional arguments for the forked JVM instance (such as memory options).
 & \\

\code{forkTimeoutSec}
 & 
How long the plugin should wait for the plugin. 0 means wait forever.
 & 
\code{0}
\\

\code{outputName}
 & 
The base name used to create the report's output file(s).
 & 
\code{index}
\\

\code{name}
 & 
The name of the report.
 & 
\code{Documentation via sphinx}
\\

\code{description}
 & 
The description of the report.
 & 
\code{Documentation via sphinx}
\\

\code{builder}
 & 
The builder to use. See the \href{http://sphinx.pocoo.org/man/sphinx-build.html?highlight=command\%20line}{Sphinx commandline documentation} for a list of possible builders.
 & 
\code{html}
\\

\code{verbose}
 & 
Whether Sphinx should generate verbose output.
 & 
\code{true}
\\

\code{warningsAsErrors}
 & 
Whether warnings should be treated as errors.
 & 
\code{false}
\\

\code{force}
 & 
Whether Sphinx should generate output for all files instead of only the changed ones.
 & 
\code{false}
\\

\code{tags}
 & 
Additional tags to pass to Sphinx. See the \href{http://sphinx.pocoo.org/markup/misc.html\#tags}{Sphinx tag documentation} for more information.
 & \\
\hline\end{tabulary}



\chapter{Building PDFs}
\label{configuration:building-pdfs}
The \code{sphinx-maven} plugin has experimental support for PDF generation. You'll turn it on
by using the pdf builder, e.g.:

\begin{Verbatim}[commandchars=\\\{\}]
\PYGZlt{}plugin\PYGZgt{}
  \PYGZlt{}groupId\PYGZgt{}org.tomdz.maven\PYGZlt{}/groupId\PYGZgt{}
  \PYGZlt{}artifactId\PYGZgt{}sphinx\PYGZhy{}maven\PYGZhy{}plugin\PYGZlt{}/artifactId\PYGZgt{}
  \PYGZlt{}version\PYGZgt{}1.0.3\PYGZlt{}/version\PYGZgt{}
  \PYGZlt{}configuration\PYGZgt{}
    \PYGZlt{}builder\PYGZgt{}pdf\PYGZlt{}/builder\PYGZgt{}
    \PYGZlt{}outputDirectory\PYGZgt{}\PYGZdl{}\PYGZob{}project.reporting.outputDirectory\PYGZcb{}/pdf\PYGZlt{}/outputDirectory\PYGZgt{}
  \PYGZlt{}/configuration\PYGZgt{}
\PYGZlt{}/plugin\PYGZgt{}
\end{Verbatim}

You'll likely also have to add some additional configuration options to your \code{conf.py}
file (usually in \code{src/site/sphinx}) to tell the pdf builder what to do. At a minimum
you'll probably need to point it to the index page by adding this to the end:

\begin{Verbatim}[commandchars=\\\{\}]
\PYG{c}{\PYGZsh{} \PYGZhy{}\PYGZhy{} Options for PDF output \PYGZhy{}\PYGZhy{}\PYGZhy{}\PYGZhy{}\PYGZhy{}\PYGZhy{}\PYGZhy{}\PYGZhy{}\PYGZhy{}\PYGZhy{}\PYGZhy{}\PYGZhy{}\PYGZhy{}\PYGZhy{}\PYGZhy{}\PYGZhy{}\PYGZhy{}\PYGZhy{}\PYGZhy{}\PYGZhy{}\PYGZhy{}\PYGZhy{}\PYGZhy{}\PYGZhy{}\PYGZhy{}\PYGZhy{}\PYGZhy{}\PYGZhy{}\PYGZhy{}\PYGZhy{}\PYGZhy{}\PYGZhy{}\PYGZhy{}\PYGZhy{}\PYGZhy{}\PYGZhy{}\PYGZhy{}\PYGZhy{}\PYGZhy{}\PYGZhy{}\PYGZhy{}\PYGZhy{}\PYGZhy{}\PYGZhy{}\PYGZhy{}\PYGZhy{}\PYGZhy{}\PYGZhy{}\PYGZhy{}\PYGZhy{}\PYGZhy{}}
\PYG{n}{pdf\PYGZus{}documents} \PYG{o}{=} \PYG{p}{[}
    \PYG{p}{(}\PYG{l+s}{\PYGZsq{}}\PYG{l+s}{index}\PYG{l+s}{\PYGZsq{}}\PYG{p}{,} \PYG{l+s}{u\PYGZsq{}}\PYG{l+s}{\PYGZlt{}file name\PYGZgt{}}\PYG{l+s}{\PYGZsq{}}\PYG{p}{,} \PYG{l+s}{u\PYGZsq{}}\PYG{l+s}{\PYGZlt{}document name\PYGZgt{}}\PYG{l+s}{\PYGZsq{}}\PYG{p}{,} \PYG{l+s}{u\PYGZsq{}}\PYG{l+s}{\PYGZlt{}author\PYGZgt{}}\PYG{l+s}{\PYGZsq{}}\PYG{p}{)}\PYG{p}{,}
\PYG{p}{]}
\end{Verbatim}

For additional options see the Sphinx section of the \href{http://lateral.netmanagers.com.ar/static/manual.pdf}{rst2pdf manual}.

Please note that alpha channels in the images (i.e. PNGs) are not supported, and will be replaced with
black pixels. This is most likely not what you want, so please don't use alpha channels in the images.


\chapter{A note on memory usage}
\label{configuration:a-note-on-memory-usage}
Sphinx is run via \href{http://www.jython.org/}{Jython} which will generate lots of small classes for various Python constructs. This means that
the plugin will use a fair amount of memory, especially PermGen space (a moderate plugin run will likely use about 80mb
of PermGen space). Therefore we suggest to either run maven with at least 256mb of heap and 128mb of PermGen space, e.g.
\begin{quote}

MAVEN\_OPTS=''-Xmx256m -XX:MaxPermSize=128m'' mvn site
\end{quote}

or use the fork parameter of the plugin, e.g.:

\begin{Verbatim}[commandchars=\\\{\}]
\PYGZlt{}plugin\PYGZgt{}
  \PYGZlt{}groupId\PYGZgt{}org.tomdz.maven\PYGZlt{}/groupId\PYGZgt{}
  \PYGZlt{}artifactId\PYGZgt{}sphinx\PYGZhy{}maven\PYGZhy{}plugin\PYGZlt{}/artifactId\PYGZgt{}
  \PYGZlt{}version\PYGZgt{}1.0.3\PYGZlt{}/version\PYGZgt{}
  \PYGZlt{}configuration\PYGZgt{}
    \PYGZlt{}fork\PYGZgt{}true\PYGZlt{}/fork\PYGZgt{}
    \PYGZlt{}argLine\PYGZgt{}\PYGZhy{}Xmx256m \PYGZhy{}XX:MaxPermSize=128m\PYGZlt{}/argLine\PYGZgt{}
  \PYGZlt{}/configuration\PYGZgt{}
\PYGZlt{}/plugin\PYGZgt{}
\end{Verbatim}


\chapter{Updating Sphinx}
\label{development:rst2pdf-bug}\label{development:updating-sphinx}\label{development::doc}
The project comes with a bash script which will update the embedded sphinx installation
automatically. Simple invoke it like so:

\begin{Verbatim}[commandchars=\\\{\}]
./src/main/build/update\PYGZhy{}sphinx.sh
\end{Verbatim}


\section{How the update script works}
\label{development:how-the-update-script-works}\begin{enumerate}
\item {} 
It sets up a temporary working directory \code{target/sphinx-tmp} and cd's into it.

\item {} 
It downloads the Jython installer for version 2.5.2 from Sourceforge.

\item {} 
It downloads the \code{ez\_setup.py} script which will setup \code{easy\_install}.

\item {} 
It installs Jython in the temporary directory.

\item {} 
It installs \code{easy\_install} in the temporary directory.

\item {} 
It uses \code{easy\_install} to install \code{docutils}, \code{pygments}, \code{jinja2}, \code{sphinx}, and \code{rst2pdf}.

\item {} 
\code{rst2pdf} depends on \code{ReportLab}, but unfortunatly that won't install
directly with \code{easy\_install}. The reason for that is that by default it is
trying to install native extensions which won't work on Jython. To workaround
that, the script downloads the \code{ReportLab} distribution directly, unpacks it,
removes the native extensions (the setup script will be fine with it), and then
runs the installation manually.

\item {} 
Both \code{reportlab} and \code{rst2pdf} have bugs with Jython (see e.g. this \href{https://code.google.com/p/rst2pdf/issues/detail?id=458}{rst2pdf bug}) which
we'll patch and then trigger Jython to pre-compile the modules again.

\item {} 
Finally, we create the \code{sphinx.jar} out of the installed modules, and move it to
\code{src/main/resources} (which will cause it to be included as a file in the plugin).

\end{enumerate}

Steps 1-8 are performed by the \emph{setup-jython-env.sh} script which is executed by
the \emph{update-sphinx.sh} script.


\chapter{Fixing bugs in one of the Python libraries}
\label{development:fixing-bugs-in-one-of-the-python-libraries}
Occasionally there are bugs in one of the python libraries, either plain bugs or bugs when running
under Jython, that need to be fixed for sphinx-maven to work. In this case, you can use the
\code{setup-jython-env.sh} script to setup an unpacked, editable sphinx jython environment:

\begin{Verbatim}[commandchars=\\\{\}]
./src/main/build/setup\PYGZhy{}jython\PYGZhy{}env.sh
\end{Verbatim}

This script will create a temporary folder \code{target/sphinx-tmp} into which it installs Jython and all
relevant libraries plus patch them as necessary (as described above).

Now you can simply use that environment directly:

\begin{Verbatim}[commandchars=\\\{\}]
\PYGZdq{}\PYGZdl{}SPHINX\PYGZus{}MAVEN\PYGZus{}DIR/target/sphinx\PYGZhy{}tmp/jython/bin/sphinx\PYGZhy{}build\PYGZdq{} \PYGZhy{}a \PYGZhy{}E \PYGZhy{}n \PYGZhy{}b pdf src/site/sphinx target/site/pdf
\end{Verbatim}

where \code{SPHINX\_MAVEN\_DIR} points to where you have checked out the sphinx maven project.

The neat thing with this is that you can now edit the python code directly. The packages are under:

\begin{Verbatim}[commandchars=\\\{\}]
\PYG{n}{target}\PYG{o}{/}\PYG{n}{sphinx}\PYG{o}{\PYGZhy{}}\PYG{n}{tmp}\PYG{o}{/}\PYG{n}{jython}\PYG{o}{/}\PYG{n}{Lib}\PYG{o}{/}\PYG{n}{site}\PYG{o}{\PYGZhy{}}\PYG{n}{packages}
\end{Verbatim}

Once you're done, simply create a patch file to an untouched sphinx + dependencies installation
(see below). E.g. for \code{rst2pdf}

\begin{Verbatim}[commandchars=\\\{\}]
diff \PYGZhy{}dur \PYGZhy{}x \PYGZdq{}*.pyc\PYGZdq{} \PYGZhy{}x \PYGZdq{}*.class\PYGZdq{} \PYGZhy{}x \PYGZdq{}requires.txt\PYGZdq{} \PYGZbs{}
    /Library/Python/2.5/site\PYGZhy{}packages/rst2pdf\PYGZhy{}0.92\PYGZhy{}py2.5.egg \PYGZbs{}
    target/sphinx\PYGZhy{}tmp/jython/Lib/site\PYGZhy{}packages/rst2pdf\PYGZhy{}0.92\PYGZhy{}py2.5.egg \PYGZbs{}
    \PYGZgt{} src/main/build/rst2pdf.patch
\end{Verbatim}

And for \code{reportlab}

\begin{Verbatim}[commandchars=\\\{\}]
diff \PYGZhy{}dur \PYGZhy{}x \PYGZdq{}*.pyc\PYGZdq{} \PYGZhy{}x \PYGZdq{}*.class\PYGZdq{} \PYGZhy{}x \PYGZdq{}hyphen.mashed\PYGZdq{} \PYGZbs{}
    /Library/Python/2.5/site\PYGZhy{}packages/reportlab\PYGZhy{}2.5\PYGZhy{}py2.5\PYGZhy{}macosx\PYGZhy{}10.7\PYGZhy{}x86\PYGZus{}64.egg/reportlab \PYGZbs{}
    target/sphinx\PYGZhy{}tmp/jython/Lib/site\PYGZhy{}packages/reportlab \PYGZbs{}
    \PYGZgt{} src/main/build/reportlab.patch
\end{Verbatim}


\chapter{Running normal Sphinx}
\label{development:running-normal-sphinx}
If you want to compare to the normal sphinx, install it like this:

\begin{Verbatim}[commandchars=\\\{\}]
easy\PYGZus{}install\PYGZhy{}2.5 sphinx rst2pdf
\end{Verbatim}

Note that this uses Python version 2.5 which is what Jython 2.x provides.
Now run Sphinx like so in your project's root directory:

\begin{Verbatim}[commandchars=\\\{\}]
sphinx\PYGZhy{}build \PYGZhy{}v \PYGZhy{}a \PYGZhy{}E \PYGZhy{}n \PYGZhy{}b html src/site/sphinx target/site
sphinx\PYGZhy{}build \PYGZhy{}a \PYGZhy{}E \PYGZhy{}n \PYGZhy{}b pdf src/site/sphinx target/site/pdf
\end{Verbatim}


\chapter{Frequently Asked Questions}
\label{faq:jekyll}\label{faq::doc}\label{faq:frequently-asked-questions}
When I put documentation generated by Sphinx on \href{http://pages.github.com/}{Github pages}, then it looks all mangled and broken. Why is that ?
\begin{quote}

Github pages by default uses a system called \href{https://github.com/mojombo/jekyll/}{Jekyll} to generate the documentation. This usually
works just fine with already generated content. However Jekyll regards directory names beginning
with an underscore to be special and won't make them available on the website. Sphinx however uses
directories like \emph{\_source} and \emph{\_static} for the site which means it won't work properly. Fortunately
the fix is easy. In your \emph{gh-pages} branch of your project on github, simply create a file called \emph{.nojekyll}
in the root folder. Once you pushed that file, Jekyll will be turned off for the site and the
documentation should look as intended.
If you are using the \href{https://github.com/github/maven-plugins}{Github maven plugins} to deploy your site, then you can tell the \emph{site-maven-plugin} to
automatically maintain the \emph{.nojekyll} file by adding:

\begin{Verbatim}[commandchars=\\\{\}]
\PYGZlt{}noJekyll\PYGZgt{}true\PYGZlt{}/noJekyll\PYGZgt{}
\end{Verbatim}

to the configuration of the \emph{site-maven-plugin} plugin (since version 0.5 of the plugin).
\end{quote}



\renewcommand{\indexname}{Index}
\printindex
\end{document}
